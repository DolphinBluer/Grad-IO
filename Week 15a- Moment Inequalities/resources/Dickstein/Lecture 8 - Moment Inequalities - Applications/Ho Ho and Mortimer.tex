%2multibyte Version: 5.50.0.2890 CodePage: 65001
%\input{tcilatex}
%\input{tcilatex}
%\input{tcilatex}
%\input{tcilatex}
%\input{tcilatex}
%\input{tcilatex}


\documentclass[notes=show]{beamer}
%%%%%%%%%%%%%%%%%%%%%%%%%%%%%%%%%%%%%%%%%%%%%%%%%%%%%%%%%%%%%%%%%%%%%%%%%%%%%%%%%%%%%%%%%%%%%%%%%%%%%%%%%%%%%%%%%%%%%%%%%%%%%%%%%%%%%%%%%%%%%%%%%%%%%%%%%%%%%%%%%%%%%%%%%%%%%%%%%%%%%%%%%%%%%%%%%%%%%%%%%%%%%%%%%%%%%%%%%%%%%%%%%%%%%%%%%%%%%%%%%%%%%%%%%%%%
\usepackage{amssymb}
\usepackage{mathpazo}
\usepackage{hyperref}
\usepackage{multimedia}

%TCIDATA{OutputFilter=LATEX.DLL}
%TCIDATA{Version=5.50.0.2890}
%TCIDATA{Codepage=65001}
%TCIDATA{<META NAME="SaveForMode" CONTENT="1">}
%TCIDATA{BibliographyScheme=Manual}
%TCIDATA{LastRevised=Friday, April 06, 2012 12:48:05}
%TCIDATA{<META NAME="GraphicsSave" CONTENT="32">}

\newenvironment{stepenumerate}{\begin{enumerate}[<+->]}{\end{enumerate}}
\newenvironment{stepitemize}{\begin{itemize}[<+->]}{\end{itemize} }
\newenvironment{stepenumeratewithalert}{\begin{enumerate}[<+-| alert@+>]}{\end{enumerate}}
\newenvironment{stepitemizewithalert}{\begin{itemize}[<+-| alert@+>]}{\end{itemize} }
\usetheme{Singapore}

%\input{tcilatex}

\begin{document}

\title{Vertical Restraints + Ho, Ho, and Mortimer (2011) Full-Line Forcing}
\author[MJ Dickstein]{Michael J. Dickstein \\
%EndAName
Stanford University}
\institute{Economics 258}
\date[3/13/13]{March 13, 2013}
\maketitle

\section{Vertical Restraints}

\begin{frame}
\frametitle{Vertical Control}

Manufacturers don't sell directly to final consumers; typically, industry is
vertically separated into upstream and downstream firms

\begin{itemize}
\item Interaction between upstream and downstream firm often involves
standard market features: price discrimination, mergers, entry, etc

\item Downstream firms can affect profits of upstream firm

\begin{itemize}
\item Determine retail price

\item Take promotional effort

\item Determine levels of inventory to hold of a product

\item Determine promotion and placement of competing products

\item Contribute technological inputs (refrigeration, etc)
\end{itemize}
\end{itemize}
\end{frame}

%--------------------------------------------------------------------------------

\begin{frame}
\frametitle{Vertical Control}

Why maintain separation?

\begin{itemize}
\item Increasing returns to sales and distribution by downstream specialists

\item Consumer demand for variety, service

\item Integration of complementary products
\end{itemize}
\end{frame}

\begin{frame}
But, in imperfectly competitive markets, there is loss from double
marginalization

\begin{itemize}
\item When the retailer and manufacturer each mark-up a good independently
of one another, there exists an externality. The resulting profits $<$
vertically-integrated profit
\end{itemize}
\end{frame}

%--------------------------------------------------------------------------------

\begin{frame}
Types of vertical restraints

\begin{itemize}
\item Exclusive territories

\begin{itemize}
\item Upstream firm assigns a single downstream firm monopoly rights to sell
in an area. See Asker (2004)
\end{itemize}

\item Exclusive dealing

\begin{itemize}
\item Downstream firm is not allowed to carry the brands of a competing
upstream (manufacturing) firm
\end{itemize}

\item Full-line forcing

\begin{itemize}
\item Dealer commits to sell all the varieties of the manufacturer's
products rather than a limited set

\begin{itemize}
\item Mixed bundling -- can buy components or bundle

\item Bundling -- can only buy the bundle, but not components
\end{itemize}
\end{itemize}
\end{itemize}
\end{frame}

%--------------------------------------------------------------------------------

\begin{frame}
Types of vertical restraints (continued)

\begin{itemize}
\item Resale price maintenance

\begin{itemize}
\item Dealer commits to a retail price or range of retail prices for the
product: minimum/maximum resale price maintenance

\item Quantity forcing or quantity rationing
\end{itemize}

\item Other Contractual arrangements

\begin{itemize}
\item Franchising, profit sharing/revenue sharing
\end{itemize}
\end{itemize}
\end{frame}

%--------------------------------------------------------------------------------

\begin{frame}
Purposes of restraints

\begin{itemize}
\item Externalities between downstream and upstream (use max RPM, quantity
forcing, full-line forcing)

\begin{itemize}
\item Use tools to avoid double marginalization problem
\end{itemize}

\item Downstream moral hazard, externalities from intrabrand competition
(exclusive territories, minimum RPM, quantity rationing)

\begin{itemize}
\item Manufacturer wants to incentivize retailer to provide service
\end{itemize}

\item Interbrand competition (exclusive dealing, full-line forcing)

\begin{itemize}
\item Manufacturer suffers when retailer sells brands other than its own
\end{itemize}
\end{itemize}
\end{frame}

%--------------------------------------------------------------------------------

\section{Ho, Ho, and Mortimer (2011)}

\begin{frame}
\frametitle{Goals of Paper}

\begin{itemize}
\item Theory Testing

\item Measurement

\item Methodology
\end{itemize}
\end{frame}

%--------------------------------------------------------------------------------

\begin{frame}
\frametitle{Goals of the Paper}

\begin{itemize}
\item Theory Testing

\begin{itemize}
\item What are the profit consequences for the manufacturer from offering
retailers full-line forcing contracts?

\item Does it reduce consumer choice or lead to higher prices?
\end{itemize}

\item Measurement

\begin{itemize}
\item Quantify how consumer demand, retailer revenues and costs, and
distributor revenues change when adding/removing FLF from contract mix.
\end{itemize}

\item Methodology

\begin{itemize}
\item Emphasizes how the frequent rotation of products in the market
provides variation in choice sets that may be helpful for identifying demand
patterns.

\item \textquotedblleft Role model\textquotedblright \ of bundling analysis:
combine detailed demand side estimation of substitution with supply side
model of firm's costs from adding inventory.
\end{itemize}
\end{itemize}
\end{frame}

%--------------------------------------------------------------------------------

\begin{frame}
\frametitle{Setting}

Innovation in recording rental transactions led to contract innovation:

\begin{itemize}
\item Linear pricing -- \$65-70 upfront fee per tape

\item Revenue sharing - \$8 upfront + 55\% of rental revenue per tape

\begin{itemize}
\item Have min and max quantity restrictions
\end{itemize}

\item FLF -- rental store purchases all titles of distributor

\begin{itemize}
\item Terms like RS, but lower up-front fee (\$3.60) and lower rev share
(retailers keep 59\%)
\end{itemize}

\item Sell-through priced (STP) titles

\begin{itemize}
\item Distributor sells tapes to consumers and retailers for \$20-25 per tape

\item Choice of this contract type not in model
\end{itemize}
\end{itemize}
\end{frame}

%--------------------------------------------------------------------------------

%--------------------------------------------------------------------------------

\begin{frame}
\frametitle{Setting}

Selection on contract type?

\begin{itemize}
\item What type of movies should retailer choose to accept under each
contract type? (Mortimer (2008))

\begin{itemize}
\item LP for high volume videos/new releases?

\item RS for niche films? RS usually have higher minimum quantity
restrictions than the avg \# of tapes bought under LP contracts.
\end{itemize}

\item What type of retailer accepts FLF contract? Blockbuster vs. small
retailer?
\end{itemize}
\end{frame}

%--------------------------------------------------------------------------------

\begin{frame}
\frametitle{Setting}

How does inventory choice affect retailer profits?

\begin{itemize}
\item Can increase retailer profits by attracting new consumers to store.
(included in costs of holding inventory)

\item High inventory may lead to high initial demand (consumers see more
tapes on shelf); can reduce later month sales (included in costs of holding
inventory)

\item Sales of substitute products fall with higher inventory on focal
product (see demand model)
\end{itemize}
\end{frame}

%--------------------------------------------------------------------------------

\begin{frame}
\frametitle{Data}

\begin{itemize}
\item Rentrak, facilitates monitoring \& payments for RS and FLF contracts
between retailers and distributors

\item 7525 retailers (excluding Blockbuster, Hollywood Video), for 1998-2001.

\item Have transaction data for 1,025 titles released: total monthly revenue
of store, zip code, size of chain, product mix detail (game, adult, rental,
and sales revenue shares)

\item Local demographics from Census by zip code

\item Title details: distributor id, month of release, genre, MPAA rating,
box office categories

\item Contract terms and \# tapes bought under each contract at
store-title-week level
\end{itemize}
\end{frame}

%--------------------------------------------------------------------------------

\begin{frame}
\frametitle{Model}

Demand: Flexible nested logit model%
\begin{eqnarray*}
u_{ijmt} &=&\delta _{jmt}+\zeta _{igmt}+(1-\sigma )\varepsilon _{ijmt} \\
\delta _{jmt} &=&\delta _{j}+\gamma _{j}z_{m}+\eta _{m}+\theta _{t}+\beta
_{t}x_{j}+\lambda _{t}c_{jm}-\alpha p_{jmt}+\xi _{jmt}
\end{eqnarray*}

\begin{itemize}
\item $(\delta _{j},\eta _{m},\theta _{t})$ - title, store,
months-since-release fixed effects

\item $p_{jmt}$ average price per rental of the tape at store $m$ in month $%
t $

\item $c_{jm}$ inventory of title $j$ at store $m$

\item $\xi _{jmt}$ unobservable quality of renting $j$ in market $m$ in
month $t$

\item Decay rate over months$,\theta _{t}$, common to genre+rating+box
office category
\end{itemize}
\end{frame}

%--------------------------------------------------------------------------------

\begin{frame}
\frametitle{Model}%
\begin{eqnarray*}
\ln (s_{jmt}) -\ln (s_{0mt})=\delta _{j}+\gamma _{j}z_{m}+\eta _{m}+\theta
_{t}+\beta _{t}x_{j} \\
+ \lambda _{t}c_{jm}-\alpha p_{jmt}+\sigma \ln (s_{jmt/gmt})+\xi _{jmt}
\end{eqnarray*}

\begin{itemize}
\item Nests: genre + box office class group. $\sigma $ is the correlation of
the idiosyncratic preferences within group.

\item Set market size to determine share of outside good.

\begin{itemize}
\item Observe only one store in a market, but there are N in phonebook.
Store gets 1/N of population (stores are exact substitutes; demand
shifts across movies within a store, not across stores).
\end{itemize}

\item Endogeneity: FLF changes composition of choice set (share of title
within its group), inventory of each title, price per rental
\end{itemize}
\end{frame}

%--------------------------------------------------------------------------------

\begin{frame}
\frametitle{Model}

Instruments

\begin{itemize}
\item Inventory IV: avg inventory of the same title across stores of the
same tier

\begin{itemize}
\item costs of taking inventory are correlated across stores of the same
size, but unobserved demand shocks in market X not related to unobserved
shocks in market Y (title and title*(area demographics) fixed
effects pick up most correlated shocks)

\item variation in IV comes from different min and max quantity restrictions
imposed by distributor
\end{itemize}
\end{itemize}
\end{frame}

%--------------------------------------------------------------------------------

\begin{frame}
\frametitle{Model}

Instruments (continued)

Share of title within group IV:

\begin{itemize}
\item (1) log of avg \# of movies of the same type (box-genre-store group)
in a month, avg over stores in same size tier.

\begin{itemize}
\item Correlated with the \# competitors to this title in the store
\end{itemize}

\item (2) avg of ln(s\_jmt/gmt) for the same title-month pair across stores
of the same tier

\begin{itemize}
\item correlated with ln(s\_jmt/gmt) in focal market

\item IV only affects shares in focal market through its effect on
ln(s\_jmt/gmt) in the focal market.
\end{itemize}
\end{itemize}

Price

\begin{itemize}
\item None. Unobservable, after fixed effects, at the store-title level.
Price changes at store-title level exogenous to demand shocks
(employees randomly change new release stickers)?
\end{itemize}
\end{frame}

%--------------------------------------------------------------------------------

\begin{frame}
\frametitle{Model}

Retailer portfolio choice problem

\begin{itemize}
\item Moment inequalities to bound the value of holding inventory (do not
model complicated retailer equilibrium strategies)

\item Intuition:

\begin{itemize}
\item In average, stores� profits from the observed portfolio of
titles and choices of inventory must exceed profits from alternative
portfolios/inventory

\item Dropping a title gives you upper bound to costs of holding inventory

\item Adding tapes (say, 10\%) provides lower bound on value of holding
inventory.
\end{itemize}

\end{itemize}
\end{frame}

%--------------------------------------------------------------------------------
\begin{frame}
\frametitle{Model}

Retailer portfolio choice problem (continued)

Procedure

\begin{itemize}
\item Calculate share of title at store m at time t using demand estimates.

\item Determine total returns to the store under its inventory constraints
(determined by the contracts it entered).

\item Subtract off payments to distributor and the costs of holding a tape
to find profits.
\end{itemize}
\end{frame}

%--------------------------------------------------------------------------------

\begin{frame}
\frametitle{Model}

Inequalities%
\[
E[\pi _{m}^{obs}(.)|I_{m}] \geq E[\pi _{m}^{altj^{\prime }}(.)|I_{m}] 
\]
\[
\pi _{m}^{obs} =\sum_{s}\sum_{j\in J_{s}}(r_{jm}^{obs}(.)-C(.)\widetilde{c}_{jm})+\eta _{m}+\rho (\widetilde{c}_{ms},k_{ms})+\varepsilon _{ms}
\]

\begin{itemize}
\item Assume $\eta _{m}$, $\rho (\widetilde{c}_{ms},k_{ms})$ difference out

\item Have instruments, $Z_{ms^{\prime }}\subset I_{m}$

\item $E[\varepsilon _{ms}|Z_{ms^{\prime }}]=0$

\item Use as IV's \{constant; indicators for size of store\}

\item Rules out error term that differs by contract type and that are
structural--that is, choice of contract depends on error.

\begin{itemize}
\item STRONG - claim is that specification of inventory holding value
captures all elements in $I_{m}$
\end{itemize}
\end{itemize}
\end{frame}

%--------------------------------------------------------------------------------

\begin{frame}
\frametitle{Model}

\begin{itemize}
\item BUT, selection effect, conditional on observed choice!

\item See Pakes (2011), Dickstein and Morales (2012)
\end{itemize}

Counterfactual experiments to infer the effect of FLF contracts on profits
of distributors

\begin{itemize}
\item Predict retailer and distributor profits under FLF

\item Predict profits were FLF not available, allowing retailer to
reoptimize their static portfolio decision

\item Compare profits under these two scenarios.
\end{itemize}
\end{frame}

%--------------------------------------------------------------------------------

\begin{frame}
\frametitle{Results}

Demand model

\begin{itemize}
\item $R^2$ of .8; model fits data well; useful for counterfactuals

\item Instrumenting for within group shares reduces $\sigma$ coefficient
(there are demand shocks that affect both within-group share and total
demand)

\item Price coefficients negative, apart from SW region
\end{itemize}

Supply model of inventory costs

\begin{itemize}
\item Find a singleton value of holding inventory

\item Larger stores and chains have a lower cost per tape

\item Store generates a positive value from holding tapes

\begin{itemize}
\item Generates revenue from selling used tapes; model might miss volume
discounts; may be value to holding variety
\end{itemize}

\item Most retailers have a higher value per tape for LP titles than RS
titles
\end{itemize}
\end{frame}

%--------------------------------------------------------------------------------

\begin{frame}
\frametitle{Results}

Counterfactuals

\begin{itemize}
\item Most distributors make profit-maximizing decisions about whether to
offer FLF (implied by supply model?)

\begin{itemize}
\item Retailers who take videos from non-FLF distributors have high take up
rates of titles without the contracts. In the counterfactual, the
distributor offers FLF with more generous prices (to get firm to take whole
line); find the distributor harmed--they could have gotten uptake without
the vertical restraint.

\item Those who offer FLF to retailers would not have attracted those
retailers were it not to offer the retailer the generous terms under FLF.
Profit maximizing for them to do so.
\end{itemize}

\item Retailers benefit from the added contract option, as long as the
default option remains.
\end{itemize}
\end{frame}

%--------------------------------------------------------------------------------

\end{document}
