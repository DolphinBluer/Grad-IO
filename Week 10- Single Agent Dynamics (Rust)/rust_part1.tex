\documentclass[xcolor=pdftex,dvipsnames,table,mathserif,aspectratio=169]{beamer}
\usetheme{metropolis}
%\usetheme{Darmstadt}
%\usepackage{times}
%\usefonttheme{structurebold}

\usepackage[english]{babel}
%\usepackage[table]{xcolor}
\usepackage{pgf,pgfarrows,pgfnodes,pgfautomata,pgfheaps}
\usepackage{amsmath,amssymb,setspace,outline}
\usepackage[latin1]{inputenc}
\usepackage[T1]{fontenc}
\usepackage{relsize}
\DeclareMathSizes{10}{10}{6}{6} 


\title [Single-agent dynamic optimization models]{Single-agent dynamic optimization models}
\author{C.Conlon - help from M. Shum and Paul Scott}
\institute{Grad IO }
\date{}
\setbeamerfont{equation}{size=\tiny}

\begin{document}
\begin{frame}
\titlepage
\end{frame}


\frame{
\frametitle{Why dynamic estimation? External validity}
\begin{itemize}
	\item<1-> Famous example: Hendel and Nevo's (2006) estimation of laundry detergent demand

	\medskip
	\item<2-> The long-run demand elasticity for laundry detergent might be zero (or very close)

	\medskip
	\item<3-> If detergent goes on sale periodically, we might see a nonzero short-run elasticity 
	(perhaps even a large one) as customers might purchase during the sales and store
	the detergent.

	\medskip
	\item<4-> Dynamic estimation typically involves estimating the primitives of decision makers' 
	objective functions. We might estimate the model using short-run variation, but once we 
	know the decision maker's objective function, we could simulate a response to long-run 
	variation.

\end{itemize}
}



\frame{
\frametitle{Why are dynamics difficult?}
\begin{itemize}
	\item The computational burden of solving dynamic problems blows up 
		as the state space gets large. With standard dynamic estimation techniques, 
		this is especially problematic, 
		for estimation may involve solving the dynamic problem many times. 
		
	\medskip
	\item Serially correlated unobservables and unobserved heterogeneity
	(easy to confuse with state dependence) 
	\medskip
	\item Modeling expectations
	\medskip	
	\item Solving for equilibria, multiplicity (dynamic games)
\end{itemize}
}


\frame{
\frametitle{Outline}
\begin{itemize}
	\item Introduction to dynamic estimation: Rust (1987)
	\item MPEC: an alternative algorithm (Su and Judd 2012)

	\medskip
	\item Conditional choice probabilities: Hotz and Miller (1993)

	\medskip
	\item Euler equation estimation: Scott (2014)

\end{itemize}
}


\section{Rust: Theory of Dynamic Discrete Choice (DDC)}

\begin{frame}{Single-agent dynamic optimization models}
Setup in Rust:
\begin{itemize}
\item Harold Zurcher manages a bus depot in Madison, WI
\item Each week he must decide to continue with the current engine $i_t=0$ and pay $c(x_{t})$ or overhaul the engine $i_t=1$ and pay fixed cost $RC$.
\item His goal is to minimize \alert{long run average cost} (discounted).
\item Buses are all  \alert{independent} of one another.
\end{itemize}
\end{frame}




\begin{frame}{Rust (1987)}
The agent makes a series of decisions $i_t,i_{t+1},\ldots \in \{0,1\}$.
\begin{eqnarray*}
\max_{\{i_1, i_2, i_3, \cdots, i_t, i_{t+1}, \cdots \}} \mathbb{E}_t \sum^{\infty}_{t=1} \beta^{t-1} \pi (x_t,i_t)\\
\pi \left(x_{t},i_{t} \right)=
		\begin{cases}
		-c\left(x_{t} \right)  & \mbox{ if }i_{t}=0\\		
		 -c\left(0\right) -RC& \mbox{ if }i_{t}=1
		\end{cases}
\end{eqnarray*}
\begin{itemize}
\item When costs are increasing in mileage $(x_{it})$ then this is an \alert{optimal stopping problem}.
\item We know from Stokey Lucas Prescott (1989) that solutions to optimal stopping problems are characterized by \alert{critical value} or \alert{cutoff rule} which we call $x^*$.
\end{itemize}
\end{frame}

\begin{frame}{Rust (1987)}
\begin{itemize}
\item For those who took first year macro, this problem is trivial. 
\item But Rust had a different goal in mind
\begin{itemize}
\item Can we use the observed decisions of the agent to identify the primitives of $\pi(x_t,i_t)$?
\item Instead of just optimizing Bellman's equation, can we actually \alert{estimate parameters}?
\end{itemize}
\item Need to make two modifications: 
\begin{itemize}
\item Parameters for $c(x_t,\theta_1)$ and $RC$
\item Full support errors $\varepsilon_{t}$, otherwise can't justify why $i(x)=1$ and $i(x')=0$ if $x'>x$ at some point in the data.
\end{itemize}
\end{itemize}
\end{frame}

\begin{frame}{Model: Modified Payoff Function}
\begin{itemize}
	\item per-period profit function:\[
		\pi \left(x_{t},i_{t}; \theta_{1}\right)=
		\begin{cases}
		-c\left(x_{t},\theta_{1}\right)+\varepsilon_{t}\left(0\right) & \mbox{ if }i_{t}=0\\		
		-\left(RC-c\left(0,\theta_{1}\right)\right)+\varepsilon_{t}\left(1\right) & \mbox{ if }i_{t}=1
		\end{cases}
		\]
	where \\
	\begin{itemize}
		\item $c\left(x_{t},\theta_{1}\right) $ -  regular maintenance costs (including expected breakdown costs),
		\item $RC$ - the net costs of replacing an engine,
		\item $\varepsilon$ - payoff shocks.
	\end{itemize}
	
	\item $x_{t}$ is the \alert{observed state variable} known to both agent and econometrician
	\item $\varepsilon$ is the \alert{unobserved state variable} known only to agent
\end{itemize}
\end{frame}

% \begin{frame}
% \frametitle{Model: Bellman}
%  Can define value function using Bellman equation:
%  \begin{align*}
% 		V_{\theta}\left(x_{t},\varepsilon_{t}\right) &= \max_{i} \left[ \pi \left(i,x_{t},\theta_1 \right) +
% 		\beta EV_{\theta}\left(x_{t},\varepsilon_{t},i \right) \right]\\
% 	 EV_{\theta}\left(x_{t},\varepsilon_{t},i_{t}\right) &=\int V_{\theta}\left(x_{t+1},\varepsilon_{t+1}\right) p\left(x_{t+1}, \varepsilon_{t+1} |x_{t},\varepsilon_{t},i_{t},\theta_{2},\theta_{3}\right)
% \end{align*}
% $EV_{\theta}$ is the \alert{ex-ante value function} (before we know which state we arrive to in $t+1$).
% \end{frame}


\begin{frame}
\frametitle{Model: Bellman}
It is helpful to define the \alert{choice-specific value function}:
\begin{align*}
\tilde V (x_t, i_t) &= \left \{ 
\begin{array}{lr}
 \pi(x_t, 1;\theta_1) + \beta \mathbb{E}\, V(0, \varepsilon ')  & \text{ if } i_t = 1 \\
 \pi(x_t , 0; \theta_1) + \beta \mathbb{E}_{x', \varepsilon ' | x_t, \varepsilon_t, i_t} V(x', \varepsilon ')   & \text{ if } i_t = 0
\end{array}
\right .\\
V(x_t,\varepsilon_t) &= \max \left\{ 
\tilde V (x_t, 1) + \varepsilon_{t}(1),
\tilde V(x_t, 0) + \varepsilon_{t}(0) 
\right\}
\end{align*}
And the \alert{ex-ante value function}
 \begin{align*}
		V_{\theta}\left(x_{t},\varepsilon_{t}\right) &= \max_{i} \left[ \pi \left(i,x_{t},\theta_1 \right) +
		\beta EV_{\theta}\left(x_{t},\varepsilon_{t},i \right) \right]\\
	 EV_{\theta}\left(x_{t},\varepsilon_{t},i_{t}\right) &=\int V_{\theta}\left(x_{t+1},\varepsilon_{t+1}\right) p\left(x_{t+1}, \varepsilon_{t+1} |x_{t},\varepsilon_{t},i_{t},\theta_{2},\theta_{3}\right)
\end{align*}
This is the period $t$ expectation of the $t+1$ period continuation value.

\end{frame}



\begin{frame}{Parameter Definitions}
\begin{itemize}
	\item $\theta_{1}$ - parameters of cost function
	\item $\theta_{2}$ - parameters of distribution of $\varepsilon$ (these will be assumed/normalized away)
	\item $\theta_{3}$ - parameters of $x$-state transition function
	\item $RC$ - replacement cost
	\item discount factor $\beta$ will be imputed (more on this later)
\end{itemize}

\end{frame}


\begin{frame}
\frametitle{Assumptions}
\footnotesize
\begin{block}{First Order Markov}\end{block}\vspace{-0.5cm}
\begin{block}{Conditional Independence Assumption}
The transition density of the controlled process $\left\{x_{t},\varepsilon_{t}\right\}$ factors as:\[
p\left(x_{t+1},\varepsilon_{t+1}|x_{t},\varepsilon_{t},i_{t},\theta_{2},\theta_{3}\right)
= q\left(\varepsilon_{t+1}|x_{t+1},\theta_{2}\right)p\left(x_{t+1}|x_{t},i_{t},\theta_{3}\right)
\]
\end{block}

\begin{itemize}
	\item CI assumption is very powerful: it means we don't have to treat $\varepsilon_{t}$ as
	a state variable, which would be very difficult since it's unobserved.
\end{itemize}
\begin{block}{IID Type I EV Assumption}
For now we will also assume that $ q\left(\varepsilon_{t+1}|x_{t+1},\theta_{2}\right) =  q\left(\varepsilon_{t+1}\right)$ and $q$ is an IID Type I Extreme Value (logit) distribution.
\end{block}
\begin{itemize}
\item This is not required for identification but commonly employed to simplify estimation.
\item Rust assumes that mean is $0$ and variance is $\pi^2/6$.
\end{itemize}
\end{frame}



\begin{frame}{ Implications}
Given the assumptions:
\begin{align*}
V_{\theta}(x_t,\varepsilon_t) &= \max \left\{ \tilde V_{\theta} (x_t, \varepsilon_t, 1) + \varepsilon_{t}(1) ,\tilde V_{\theta} (x_t, \varepsilon_t, 0) + \varepsilon_{t}(0) \right\}\\
Pr(i_t=1 | x_t,\varepsilon_t,\theta) &= Pr\left(\varepsilon_{t}(1) - \varepsilon_{t}(0) \geq \tilde V_{\theta} (x_t, \varepsilon_t, 0) - \tilde V_{\theta} (x_t, \varepsilon_t, 1) \right) \\
&= \frac {\exp[\tilde V_{\theta} (x_t, \varepsilon_t, 1)]  }{\exp[\tilde V_{\theta} (x_t, \varepsilon_t, 0)] + \exp[\tilde V_{\theta} (x_t, \varepsilon_t, 1)]} 
\end{align*}
This expression is logit-like. Recall the static logit:
\begin{align*}
Pr(i_t=1 | x_t,\varepsilon_t,\theta) &= \frac {\exp[u_{\theta} (x_t, 1)]  }{\exp[u_{\theta} (x_t, 0)] + \exp[u_{\theta} (x_t, 1)]} 
\end{align*}
\end{frame}


\begin{frame}{Theorem 1 preview}
\begin{itemize}
	\item Assumption CI has two powerful implications:
	\begin{itemize}
		\medskip
		\item We can write $ EV_{\theta}\left(x_{t},i_{t}\right)$ instead of $EV_{\theta}\left(x_{t},\varepsilon_{t},i_{t}\right)$,
		\medskip
		\item We can consider a Bellman equation for $ V_{\theta}\left(x_{t}\right)$, which is computationally simpler than
		the Bellman equation for $V_{\theta}\left(x_{t},\varepsilon_{t}\right) $.
	\end{itemize}
\end{itemize}
\end{frame}

\begin{frame}{Theorem 1}
\footnotesize
\begin{block}{Theorem 1}
Given CI,
\begin{align*}
	P\left(i|x,\theta\right) = \frac{\partial}{\partial \pi \left(x,i,\theta_{1}\right)} W\left( \pi \left(x,\theta_{1}\right) +\beta EV_{\theta}\left(x\right) |x,\theta_{2}\right)
\end{align*}
and  $EV_{\theta}$ is the unique fixed point of the contraction mapping:
\begin{align*} 
	EV_{\theta}\left(x,i\right) = \int_{y} W\left(\pi \left(y,\theta_{1}\right)+
	\beta EV_{\theta}\left(y\right)|y,\theta_{2}\right)
	p\left(dy|x,i,\theta_{3}\right)
\end{align*}
\begin{itemize}
	\item $P\left(i|x,\theta\right)$ is the probability of action $i$ conditional on state $x$
	\item $W\left( \cdot |x,\theta_{2}\right)$ is the (ex-ante) surplus function: 
	\begin{align*}
	W\left(v |x,\theta_{2}\right) \equiv \int_{\varepsilon}  \max_{i} \left[v\left( i \right) +\varepsilon\left( i \right)\right] 
	q\left(d\varepsilon|x,\theta_{2}\right)	
	\end{align*}
\end{itemize}
\end{block}
\end{frame}


\begin{frame}{Theorem 1 example: logit shocks}
\footnotesize
\begin{itemize}
	\item $v_{\theta}\left(x,i\right) \equiv \pi \left(x,i,\theta_{1}\right)+\beta EV_{\theta}\left(x,i\right)$ -- the \alert{choice-specific value function}.

	\smallskip
	\item Suppose that $\varepsilon\left(i\right)$ is distributed independenly
	across $i$ with $Pr\left(\varepsilon\left(i\right)\le\varepsilon_{0}\right)=e^{-e^{-\varepsilon_{0}}}$ -- logit shocks.
	Then,
	\begin{align*}
	\begin{array}{ccl}
	W\left(v\left(x\right)\right) & =  
	\int\max_{i}\left[v\left(x,i\right)+\varepsilon\left(i\right)\right]
	\prod_{i}e^{-\varepsilon\left(i\right)}e^{-e^{-\varepsilon\left(i\right)}}d\varepsilon\\
	\\
	 & =  \ln\left(\sum_{i}\exp\left(v\left(x,i\right)\right)\right)+\gamma
	\end{array}
	\end{align*}
	where $\gamma\approx .577216$ is Euler's gamma.

	\smallskip
	\item It is then easy to derive expressions for conditional choice probabilities:
	\begin{align*}
	P\left(i|x,\theta\right)= \frac{\exp\left(v_{\theta}\left(x,i\right)\right)}{\sum_{i'}\exp\left(v_{\theta}\left(x,i'\right)\right)}
	\end{align*}
	\smallskip
	\item The conditional value function plays the same role as a static utility function
	when computing choice probabilities.
\end{itemize}
\end{frame}


\begin{frame}{Is this testable?}
Problem: How do we do know true model is dynamic and not static logit?
\begin{itemize}
\item The general problem is that we can often write an observationally equivalent \alert{myopic model}
\item In general without additional (parametric) restrictions we are under identified (especially w.r.t $\beta$).
\item May be able to determine ``reasonableness'' of parameters (how much could it cost to replace a bus engine?)
\item Magnac Thesmar (Ecma 2002): we need \alert{exclusion restrictions}:
\begin{itemize}
\item Variables that move $u_{\theta}(x)$ but not $EV_{\theta}(x,i)$.
\item Variables that move $EV_{\theta}(x,i)$ but not $u_{\theta}(x)$.
\end{itemize}
\end{itemize}
\end{frame}





\end{document}