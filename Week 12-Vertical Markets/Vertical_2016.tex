\documentclass[xcolor=pdftex,dvipsnames,table,mathserif]{beamer}
\usetheme{default}
%\usetheme{Darmstadt}
%\usepackage{times}
%\usefonttheme{structurebold}

\usepackage[english]{babel}
%\usepackage[table]{xcolor}
\usepackage{pgf,pgfarrows,pgfnodes,pgfautomata,pgfheaps}
\usepackage{amsmath,amssymb,setspace}
\usepackage[latin1]{inputenc}
\usepackage[T1]{fontenc}
\usepackage{relsize}
\usepackage[absolute,overlay]{textpos} 
\newenvironment{reference}[2]{% 
  \begin{textblock*}{\textwidth}(#1,#2) 
      \footnotesize\it\bgroup\color{red!50!black}}{\egroup\end{textblock*}} 

\DeclareMathSizes{10}{10}{6}{6} 


\title{Vertical Control (Undergrad Notes)}
\author{Chris Conlon}
%\institute{EC853}
\date{\today}


%%%%%%%%%%%%%%%%%%%%%%%%%%%%%%%%%%%%%%%%%%%%%%%%%%
%%%%%%%%%%%%%%%%%%%%%%%%%%%%%%%%%%%%%%%%%%%%%%%%%%%

\begin{document}
% outline at the beginning of each section
%\AtBeginSection[]
%{
%\begin{frame}[plain]
% \frametitle{Outline}
% \tableofcontents[currentsection]
% \normalsize
%\end{frame}
%}

\frame[plain]{\titlepage}

\section[Outline]{}
\frame[plain]{
\frametitle{Outline}
\tableofcontents}

\section{Overview}


\frame[plain]{ 
\frametitle{Vertical Control}

Manufacturers rarely supply final consumers directly (as we have modeled them so far).  
Instead, most industries are vertically separated.\\  
\vspace{0.2in} 
We often refer to firms in these markets as upstream and downstream firms.  In these settings, 
downstream firms are the customers of the upstream firms, and many of the standard issues 
still apply.  For example:\\
\begin{enumerate}
\item choice of price is endogenous\\
\item price discrimination (both the upstream and downstream firms)\\
\item mergers\\
\item entry, etc.\\ 
\end{enumerate}
}

\frame[plain]{
\frametitle{Vertical Control}

However, things can also get more complicated in vertically separated environments.  In particular, downstream firms 
do not usually consume the good, but typically make further decisions regarding the product.\\ 
\vspace{0.2in}
Examples of activities of downstream firms:\\
\begin{enumerate}
\item determination of final price\\
\item promotional effort\\
\item placement of product on store shelves\\
\item promotion and placement of competing products\\
\item technological inputs\\
\end{enumerate}
}

\frame[plain]{
\frametitle{Vertical Control}
Why don't manufacturers simply engage in direct marketing to consumers?\\
\vspace{0.2in}
Some reasons:\\
\begin{enumerate}
\item increasing returns to distribution due to shopping needs or travel costs for consumers\\
\item choice of variety\\
\item demand for service \\
\item integration of complementary products\\
\item different geographical markets, etc. 
\end{enumerate}
}

\frame[plain]{
\frametitle{Vertical Control}
Unlike the consumption activities of final consumers, the activities of the downstream firms may affect the profits of the upstream firm.\\
\vspace{0.2in}
This is why upstream firms care about the activities of the downstream firms, and why we study vertical control/restraints between firms in these settings.\\
\vspace{0.2in}
We focus on the incentives for vertical control when the market for the intermediate good is imperfectly competitive.\\
}
\frame[plain]{
\frametitle{Vertical Control}
A common benchmark for what firms can achieve through vertical control is the ``vertically integrated profit.''  This is the maximum industry or aggregate (manufacturer plus retailer) profit.  \\
\vspace{0.2in}
If firms use vertical restraints efficiently, they should achieve the vertically integrated profit.\\
}

\frame[plain]{
\frametitle{Vertical Control}

There are several types of vertical restraints used by firms in vertically-separated markets:\\
\begin{enumerate}
\item {\it Exclusive Territories}: a dealer/ distributor/ retailer is assigned a (usually geographic) territory by the manufacturer/ upstream firm and given monopoly rights to sell in that area.\\
\vspace{0.2in}
\item {\it Exclusive Dealing}:  a dealer/ distributor/ retailer is not allowed to carry the brands of a competing upstream firm.\\
\vspace{0.2in}
\item {\it Full-line forcing}:  a dealer is committed to sell all the varieties of the manufacturer's products rather than a limited selection.  (i.e., the upstream firm ties all its products to sell to the downstream firm).\\
\end{enumerate}
}

\frame[plain]{
\frametitle{Vertical Control}
\begin{enumerate}
\setcounter{enumi}{3}
\item {\it Resale Price Maintenance}:  a dealer commits to a retail price or a range of retail prices for the product.  This can 
take the form of either minimum resale price maintenance or maximum resale price maintenance.  Equivalently, firms can engage in quantity forcing or quantity rationing.\\
\vspace{0.2in}
\item {\it Contractual arrangements}:  upstream and downstream firms write contracts to provide greater flexibility in the transfer 
of the product.  Profit sharing and revenue sharing are the most common, which we'll see soon.  Also, franchising arrangements.\\
\end{enumerate}
}


\frame[plain]{
\frametitle{Vertical Control}
The typical outline of vertical control is as follows:\\
\begin{enumerate}
\item Basic Framework\\

\item Externalities between downstream and upstream firms (Maximum Resale Price Maintenance, Quantity Forcing, Contractual Arrangements, or Full-line Forcing)\\

\item Downstream Moral Hazard, or Externalities from Intrabrand competition (Exclusive Territories, Minimum RPM, or Quantity Rationing)\\

\item Interbrand competition (Exclusive Dealing or possibly Full-line Forcing)\\
\end{enumerate}

}
\section{Basic Framework}

\frame[plain]{
\frametitle{Basic Framework}
\begin{itemize}
\item Simple model: homogeneous good with (inverse) demand given by:
\[ p = a - Q\]
\item Suppose we have a monopolistic manufacturer and we have given exclusive rights to a dealer to sell the product of the manufacturer, so both the upstream and downstream firms are monopolistic. \\ 
\vspace{0.1in}
\item The downstream firm has marginal cost of selling 
the product of $d$ which is equal to the wholesale cost of purchasing the product from the manufacturer.\\
\vspace{0.1in}
\item The manufacturer has marginal cost of producing the good equal to $c$.
\end{itemize}
}

\frame[plain]{
\frametitle{Basic Framework}
Dealer maximizes his profit given by
\[ \pi_d = p(Q) Q - d Q = (a - Q) Q - d Q\]


F.O.C.:
\[ \frac{\partial \pi_d}{\partial Q} = 0 = a - 2Q - d\]

\[Q^* = \frac{a - d}{2} \hspace{0.4in} p^* = \frac{a+d}{2} \hspace{0.4in} \pi_d = \frac{(a - d)^2}{4}\]\\
\vspace{0.1in}
Next solve for the upstream firm's choice of $d$.
}


\frame[plain]{
\frametitle{Basic Framework}
Manufacturer maximizes profit given by
\[\pi_m = (d - c) Q = (d - c) \frac{a - d}{2}\]

F.O.C.:
\[ \frac{\partial \pi_m}{\partial d} = 0 = a - 2d + c\]

\[d^* = \frac{a + c}{2} \hspace{0.4in} \pi_m = \frac{(a - c)^2}{8}\]\\
\vspace{0.1in}
Note that we can now substitute into the dealer's solutions (for $d$) and get:
\[Q^* = \frac{a - c}{4} \hspace{0.4in} p^* = \frac{3a+c}{4} \hspace{0.4in} \pi = \frac{(a - c)^2}{16}\]


}

\section{Externalities}

\frame[plain]{
\frametitle{Externalities}
Results:\\
\begin{enumerate}
\item The manufacturer earns a higher profit than the dealer\\
\item The manufacturer could earn a higher profit if he did the selling himself.\\
\end{enumerate}
Total industry profit in this case is lower 
than the vertically integrated profit.  Shown here:

\[ \pi_{VI} = \frac{(a - c)^2}{4} > (\pi_d + \pi_m) = \frac{3 (a - c)^2}{16}\]

The presence of two markups introduces an inefficiency.  

}

\frame[plain]{
\frametitle{Externalities}
This basic fact is called:\\
\begin{itemize}
\item double-monopoly markup problem, 
\item successive monopolies problem, or 
\item double marginalization.
\end{itemize}
\vspace{0.1in}
As mentioned earlier, there are many ways around these problems, including RPM, contracts, etc. \\
\vspace{0.2in}
There are also other problems that arise, and sometimes firms might choose to create a successive monopoly problem in order to solve other incentive problems in 
the vertical channel.  
}

\frame[plain]{
\frametitle{Externalities}



\begin{picture}(5,35)(-40,-5)
\setlength{\unitlength}{1mm}

\linethickness{1pt}
% this one is horizontal
\put(-5,-35){\line(1,0){85}}
% this one is vertical
\put(-5,-35){\line(0,1){70}}
%labels
\put(-10,35){\makebox(0,0)[l]{$P$}}
\put(85,-35){\makebox(0,0)[r]{$Q$}}

%this one is demand
\put(-5,25){\line(4,-3){85}}
\put(80,-41){\makebox(0,0)[r]{$D_r$}}
%this one is marginal revenue
\put(-5,25){\line(2,-3){42.5}}
\put(45,-41){\makebox(0,0)[r]{$MR_r=D_w$}}
%this one is marginal revenue #2
\put(-5,25){\line(1,-3){21.25}}
\put(20,-41){\makebox(0,0)[r]{$MR_w$}}
% this one is marginal cost
\put(-5,-25){\line(2,0){70}}
\put(72,-25){\makebox(0,0)[r]{$MC$}}
% this one is wholesale price
%\put(-5,0){\line(2,0){33.5}}
%%%%%
\qbezier[13](-5,0)(3.5,0)(12,0)
%\put(70,0){\makebox(0,0)[r]{\tiny{$wholesale \, price$}}}
\put(-11,0){\makebox(0,0)[l]{$P_w$}}


%this one is VI choice of q
%%\qbezier[25](28,5,-35)(28.5,-17.5)(28.5,0)
%this is 2nd monopoly choice of q
\qbezier[25](12,-25)(12,-8.75)(12,12.5)
%this is retail price
\qbezier[13](-5,12.5)(3.5,12.5)(12,12.5)
\put(-10,12.5){\makebox(0,0)[l]{$P_r$}}

\linethickness{1pt}
% this one is horizontal
%%\put(-5,-35){\line(2,0){85}}
% this one is vertical
%%\put(-5,-35){\line(0,2){70}}
%labels
%%\put(-10,30){\makebox(0,0)[l]{\tiny{$P$}}}
%%\put(85,-35){\makebox(0,0)[r]{\tiny{$Q$}}}

%this one is demand
%%\put(-5,25){\line(4,-3){85}}
%%\put(80,-40){\makebox(0,0)[r]{\tiny{$D_r$}}}
%this one is marginal revenue
%%\put(-5,25){\line(2,-3){42.5}}
%%\put(45,-40){\makebox(0,0)[r]{\tiny{$MR_r=D_w$}}}
%this one is marginal revenue #2
%%\put(-5,25){\line(1,-3){22.25}}
%%\put(20,-40){\makebox(0,0)[r]{\tiny{$MR_w$}}}
% this one is marginal cost
%%\put(-5,-25){\line(2,0){140}}
%%\put(70,-25){\makebox(0,0)[r]{\tiny{$MC$}}}
% this one is wholesale price
%\put(-5,0){\line(2,0){33.5}}
\qbezier[25](-5,0)(12,0)(28.5,0)
%\put(70,0){\makebox(0,0)[r]{\tiny{$wholesale \, price$}}}
%%\put(-10,0){\makebox(0,0)[l]{\tiny{$P_w$}}}


%this one is VI choice of q
\qbezier[25](28.5,-35)(28.5,-17.5)(28.5,0)
%this is 2nd monopoly choice of q
%%\qbezier[50](12,-25)(12,-8.75)(12,12.5)
%this is retail price
%%\qbezier[25](-5,12.5)(3.5,12.5)(12,12.5)
%%\put(-10,12.5){\makebox(0,0)[l]{\tiny{$P_r$}}}

\end{picture}

}

\frame[plain]{
\frametitle{Externalities}
\begin{itemize}
\item {\it Maximum Resale Price Maintenance (Maximum RPM)}, or {\it Quantity Forcing}: \\Set a maximum resale price below the optimal retail price, in order to mitigate the double marginalization problem.  Equivalently, use a quantity forcing arrangement.\\
\vspace{0.2in}
Examples include:
\begin{itemize}
\item gasoline
\item newspapers
\item ``suggested retail prices''
\end{itemize}
\vspace{0.2in}
Important court cases are:
\begin{itemize}
\item Albrecht v. The Herald Co. (1968) (per se)
\item State Oil Co. v. Khan (1997) (rule of reason)
\end{itemize}
\end{itemize}
}

\frame[plain]{
\frametitle{Externalities}
\begin{itemize}
\item {\it Contractual Arrangements}: \\Instead of using Maximum RPM, write other types of contracts.  Perhaps lease the good to the downstream firm, perhaps use 
profit-sharing contracts.  \\
\vspace{0.1in}
\item {\it Profit-sharing or revenue-sharing contracts}: \\Similar to a two-part tariff.  Instead of charging linear prices, the
manufacturer requires a lump-sum transfer as well as a per-unit charge.
\end{itemize}
} 


\frame[plain]{
\frametitle{Externalities}
Consider a simple linear demand model in a vertical setting where consumer demand is, for example,

\[Q = V - \eta p \]

Where $p$ is the price charged under simple linear pricing with no cover charge.  With two monopolists, we end up with:\\

Demand facing manufacturer, optimal wholesale price, optimal retail price:

\[ Q = V - 2\eta p, \hspace{0.4in} p_w^* = \frac{V}{4\eta} \hspace{0.4in} p_r^* = \frac{5V}{8\eta}\]

}

\frame[plain]{
\frametitle{Externalities}

And firm profits under the case of linear pricing only are:

\[ \pi_m = \frac{V}{4}(\frac{V}{\eta} - \frac{V}{4\eta}) = \frac{3V^2}{16\eta} \]

\[ \pi_r = \frac{3V}{8}(\frac{V}{\eta} - \frac{5V}{8\eta}) = \frac{9V^2}{64\eta} \]

\[(\pi_r + \pi_m = \frac{21V^2}{64\eta}) \]

(This gives us well-behaved profit functions that actually have a maximum so we don't require capacity constraints.  Note that $p_r^*$ accounted for the fact that the downstream firm paid a marginal cost of $p_w^*$.)  

}

\frame[plain]{
\frametitle{Externalities}
Now case 2, with a lump-sum payment:  (Remember we're assuming that production costs are zero.)\\
\vspace{0.2in}
The best the downstream firm can do when he does not have to pay for the incremental units is to maximize profit according to:

\[ \max \pi_r = p (V - \eta p)\]

Which yields 

\[ p_r^* = \frac{V}{2\eta} \hspace{0.4in}  \pi_r = \frac{V^2}{2\eta} > \frac{21V^2}{64\eta}\]


}

\frame[plain]{
\frametitle{Externalities}
Let's say the upstream firm charges the downstream firm a lump-sum payment $C$ for quantity $Q^*$ of the good.\\
The upstream firm's profit function is $\pi_m = C$.\\
\vspace{0.2in}
Satisfy a zero profit condition for the downstream firm.  Then 

\[ \pi_m = C \mbox{ s.t. } \frac{V^2}{2\eta} - C = 0 \]

Therefore,

\[ C = \frac{V^2}{2\eta}\]

The upstream firm makes profits of $\frac{V^2}{2\eta}$ and the downstream firm makes zero profit, but aggregate profits are 
higher.

}

\frame[plain]{
\frametitle{Externalities}
We'll see that sometimes firms face a menu of such contracts.  In that case, the firms will bargain over the extra 
profit (relative to the baseline profits under linear per-unit pricing).\\
\vspace{0.2in}  
The upstream firm will have to give at least $\frac{3V^2}{16\eta}$ to the downstream firm in order to induce him to accept the other contract.

 }
\section{Downstream Moral Hazard}

\frame[plain]{
\frametitle{Downstream Moral Hazard, or Externalities from Intrabrand Competition}

Consider the example of promotions or advertising.  Assume (inverse) demand is given by

\[ p = \sqrt A - Q\]

The manufacturer sells to two dealers who compete in price.  Denote the wholesale price as $d$ and advertising expenditures as 
$A_1$ and $A_2$, where $A = A_1 + A_2.$  \\
\vspace{0.2in}
First result:
For any given $d$, no dealer will engage in advertising and demand would shrink to zero, with no sales.  (Firms compete in price, and they sell a homogeneous product.)

}

\frame[plain]{
\frametitle{Downstream Moral Hazard, or Externalities from Intrabrand Competition}
What can Minimum Resale Price Maintenance (or Quantity Rationing) do?\\
\vspace{0.2in}

{\it Minimum} Resale Price Maintenance: $p= p^f \geq d$\\
\vspace{0.2in}
Now demand is:

\[ Q = \sqrt{(A_1 + A_2)} - p^f\]
}

\frame[plain]{
\frametitle{Downstream Moral Hazard, or Externalities from Intrabrand Competition}
Assume that quantity demanded is split evenly between the two retailers.  The only strategic variable for the retailers is 
$A$.  Thus, writing profits as a function of $A$ and finding the F.O.C. yields:\\



\[ \pi_i = \frac{\sqrt{(A_i + A_j)} - p^f}{2} (p^f - d) - A_i\]

F.O.C.:

\[ 0 = \frac{\partial \pi_i}{\partial A_i} = \frac{p^f - d}{4\sqrt{(A_i + A_j)}} - 1 \]
}

\frame[plain]{
\frametitle{Downstream Moral Hazard, or Externalities from Intrabrand Competition}
Note that we can only identify the sum of $A_1 + A_2$ and not $A_1$ and $A_2$ individually. \\
\vspace{0.2in}
But the idea is that retailers will compete on promotion now.  As long as $p^f > d$ then at least one retailer has an incentive to advertise, and the totaldollars spent on ads increases with the markup.\\
\vspace{0.2in}
Thus, we can induce dealers or retailers to allocate resources for promoting the product, or exert other forms of effort in distributing the product.  
}

\frame[plain]{
\frametitle{Downstream Moral Hazard, or Externalities from Intrabrand Competition}

Examples of minimum RPM, also sometimes called\\ ``Telser special services'': 
\begin{itemize}
\item perfume 
\item cameras 
\item Coors beer 
\item Windows 98, Windows XP, Vista
\item books
\item many, many retail products (toys, electronics, etc.)\\
\end{itemize}
\vspace{0.2in}
Important court cases include:\\
\begin{itemize}
\item Miles Medical v. John Park and Sons (1911) (per se)\\
\item Leegin Creative Leather Products v. PSKS (2007) (rule of reason)\\
\end{itemize}

}

\frame[plain]{
\frametitle{Downstream Moral Hazard, or Externalities from Intrabrand Competition}
Note that the problem of downstream moral hazard was that competition between the retailers resulted in too much 
competition downstream, so that firms could not afford to advertise as a vertically-integrated firm would choose to do.\\
\vspace{0.2in}
Another (non-price) way around that: Exclusive Territories or ``Territorial Dealerships''\\

}

\frame[plain]{
\frametitle{Downstream Moral Hazard, or Externalities from Intrabrand Competition}

Simple Model of Exclusive Territories:\\
\begin{itemize}
\item Manufacturer must choose whether to grant dealerships to several or one dealer(s).
\item A similar literature looks at these decisions in the context of licensing for inventors. (i.e., when I invent a new product, do I want to sell it myself, license it exclusively to a retailer/distributor, or license it to many competing retailers?)
\end{itemize}
}

\frame[plain]{
\frametitle{Downstream Moral Hazard, or Externalities from Intrabrand Competition}

Model:\\
\vspace{0.2in}
Assume mfg's production cost c = 0\\
The wholesale (linear) price is d (strategy variable of the mfg)\\
There is a fixed cost to setting up a dealership given by F $>$ 0\\
\vspace{0.2in}
Consider an `address' model of differentiated products with consumers located along a line:

\begin{picture}(15,15)(25,20)
\setlength{\unitlength}{.7mm} \linethickness{1pt}
\put(35,0){\line(2,0){100}} \put(35,-5){\line(0,2){10}}
\put(135,-5){\line(0,2){10}}
\put(35,10){\makebox(0,0)[c]{\small{Consumer 1}}}
\put(135,10){\makebox(0,0)[c]{\small{Consumer 2}}}
\end{picture}

\vfill 
\vspace{0.2in}
Here, consider 2 consumers located at the edge of town.
}

\frame[plain]{
\frametitle{Downstream Moral Hazard, or Externalities from Intrabrand Competition}


The manufacturer must decide whether to license a single dealer in
the center of town:\\
\vspace{0.5in}
\begin{picture}(15,15)(25,-20)
\setlength{\unitlength}{.7mm} \linethickness{1pt}
\put(35,0){\line(2,0){100}} \put(35,-5){\line(0,2){10}}
\put(135,-5){\line(0,2){10}} \put(35,10){\makebox(0,0)[c]{\small{C
1}}} \put(135,10){\makebox(0,0)[c]{\small{C 2}}}
%
\put(85,-5){\line(0,2){10}}
\put(85,10){\makebox(0,0)[c]{\small{Dealer}}}
\put(135,4){\vector(-2,0){48}}
\put(110,7){\makebox(0,0)[c]{\small{T}}}
\end{picture}

Or two dealerships at the edges of town:\\

\begin{picture}(15,15)(25,20)
\setlength{\unitlength}{.7mm} \linethickness{1pt}
\put(35,0){\line(2,0){100}} \put(35,-5){\line(0,2){10}}
\put(135,-5){\line(0,2){10}} \put(35,10){\makebox(0,0)[c]{\small{C
1}}} \put(135,10){\makebox(0,0)[c]{\small{C 2}}}
%
\put(35,-12){\makebox(0,0)[c]{\small{Dealer 1}}}
\put(135,-12){\makebox(0,0)[c]{\small{Dealer 2}}}
\put(135,4){\vector(-2,0){98}}
\put(85,7){\makebox(0,0)[c]{\small{2T}}}
\end{picture}
}

\frame[plain]{
\frametitle{Downstream Moral Hazard, or Externalities from Intrabrand Competition}


Assume the travel costs for a consumer of
travelling from either
edge of town to the center are T.\\
\vspace{.2in}
Also assume that consumers have the same basic value for the
product, = B.\\
\vspace{.2in}
Thus, the utility functions of the consumers are:\\
\[ u^i = \left\{ \begin{array}{ll}
B - T - p & \mbox{buy from a center dealer}\\
B - p_i & \mbox{buy from local (edge) dealer}\\
B - 2T - p_j & \mbox{buy from other side of town}\\
0 & o.w.\\
\end{array} \right. \]
}


\frame[plain]{
\frametitle{Downstream Moral Hazard, or Externalities from Intrabrand Competition}


Let's work out profits in each case.\\
\vspace{.2in}
Dealer(s) choose $p$ (retail price to consumers)\\
Manufacturer chooses $d$ (wholesale price to dealers)\\
\vspace{0.2in}
\underline{Exclusive `Town-center' Dealer}\\
Dealer's task is easy: extract all consumer surplus\\
\hspace*{0.2in} $\Rightarrow$ p = B - T\\
\hspace*{0.2in} Q = 2\\
$\hspace*{0.2in} \pi_d$ = 2(p - d) = 2(B - T - d) - F\\
\hspace*{0.2in} (don't forget fixed cost)\\
\hspace*{0.2in} Mfg's problem -- \LARGE $\ast$ \\
}


\frame[plain]{
\frametitle{Downstream Moral Hazard, or Externalities from Intrabrand Competition}
\underline{Two Edge--of--Town Dealerships}\\
\vspace{.2in}
2 cases: large town, and small town\\
(Really these are towns with high or low transportation costs, or
possibly high/low search costs for consumers\ldots)\\
\vspace{.2in}
\underline{Case I: Large Town}\\
\vspace{.2in}
Define a large town as one in which T $> \frac{F}{4}$\\
\vspace{.2in}
Define a small town as one in which T $< \frac{F}{4}$\\
}

\frame[plain]{
\frametitle{Downstream Moral Hazard, or Externalities from Intrabrand Competition}


\LARGE $\ast$ \normalsize Manufacturer's problem in the case of 'Town
Center' Dealer:\\
\vspace{.2in}
\hspace*{0.2in} max d\ \ $\pi_m$ = d$\cdot$ Q\\
\vspace{.2in}
\hspace*{0.2in} Subject to $\pi_d$ = 2(B - T - d) - F $\geq$ 0 \hspace{.5in} (1)\\
\vspace{.1in}
\hspace*{0.2in} (Zero-profit condition for the dealer).\\
\hspace*{0.2in} Suppose this constraint is satisfied exactly.  Then:\\
\vspace{.2in}
\hspace*{0.2in} d* = B - T - $\frac{F}{2}$\ \ (rearranging (1))\\
\vspace{.2in}
\hspace*{0.2in} $\Rightarrow$ \hspace{0.2in} $\pi_m$ = (B - T -
$\frac{F}{2})\cdot$ 2\\
\vspace{.2in}
\hspace*{0.2in} \fbox{$\pi_m$ = 2(B - T) - F}
}

\frame[plain]{
\frametitle{Downstream Moral Hazard, or Externalities from Intrabrand Competition}


We want to find an equilibrium in which the two dealers do
not undercut each others' prices.  Thus, we want to find an
equilibrium s.t.\\
\vspace{.2in}
$\pi_1 = p_1$ -d - F $\geq$ 2(($p_2$ - 2T) -d) - F\\
(Profit to firm 1 is higher if they don't undercut
firm 2 and take all of firm 2's customers), and (vice versa).\\
\vspace{.2in}
$\pi_2 = p_2$ - d - F $\geq$ 2($p_1$ - 2T - d) - F\\
(same requirement for firm 2)\\
\vspace{.3in}
}

\frame[plain]{
\frametitle{Downstream Moral Hazard, or Externalities from Intrabrand Competition}


\emph{Large town} (T $> \frac{F}{4}$)\\
\vspace{.2in}
In this case, subsidizing the consumers' transportation cost is
too expensive (you have to reduce your price by a
\underline{lot}). So we can support the equilibrium above.\\
\vspace{.2in}
Thus, consumers have w.t.p. = B (i.e., utility = B - p)\\
\vspace{0.2in}
Dealer charges p = B, extracting all consumer surplus.\\
\vspace{0.2in}
Dealer profits are p - d - F = $\pi$\\
\vspace{.2in}
Therefore\ \fbox{$\pi$ = B - d - F}
}

\frame[plain]{
\frametitle{Downstream Moral Hazard, or Externalities from Intrabrand Competition}


What is the optimal wholesale price (d) for the manufacturer to
charge?\\
\vspace{.2in}
Dealer will stay in business as long as $\pi \geq$ 0.  So, if mfg.
changes d = B - F, dealer makes:\\
\begin{center}
\begin{tabbing}
$\pi$ \== B - d - F\\
\>= B - (B - F) - F\\
\>= 0
\end{tabbing}
\end{center}
Manufacturer extracts all rents by setting d = B - F\\
(i.e., mfg extracts all the consumer's w.t.p. from the dealer, but
allows the dealer to cover fixed costs)\\
\vspace{.2in}
}

\frame[plain]{
\frametitle{Downstream Moral Hazard, or Externalities from Intrabrand Competition}

By the way, check that this is an undercut-proof equilibrium!\\
$p_1$ - d - F $\geq$ 2($p_2$ - 2T - d) - F, and\\
$p_2$ - d -F $\geq$ 2($p_1$ - 2T - d) - F\\
\vspace{0.2in}
 Notice that\\
$p_1$ - d - F $\geq$ 2($p_2$ - 2T - d) - F\\
$\Rightarrow$ B - B + F - F $\geq$ 2(B - 2T - B + F) - F\\
\vspace{.1in}
which holds if\\
\vspace{.1in}
0 $\geq$ 2B - 4T - 2B + 2F - F\\
\vspace{.1in}
$\Rightarrow$ 4T $\geq$ F, or\ \ T $\geq
\frac{F}{4}$\\
}

\frame[plain]{
\frametitle{Downstream Moral Hazard, or Externalities from Intrabrand Competition}

Finally, profits of the mfg. are:\\
$\pi_m$ = 2d\\
\hspace*{.2in} = 2(B - F)\\
\vspace{.2in}
What should the mfg. do?  Set up 1 dealership or two?\\
$\pi_m$ of 1 dealership = 2(B - T) - F\\
$\pi_m$ of 2 dealerships = 2(B - F)\\
\vspace{.2in}
Therefore, a single dealership is more profitable if\\
2B - 2T - F $>$ 2B - 2F\\
$\Rightarrow$ \fbox{F $>$ 2T}\\
And we're in a large town, so that\\
\fbox{F $<$ 4T}
}

\frame[plain]{
\frametitle{Downstream Moral Hazard, or Externalities from Intrabrand Competition}
\emph{Large Town: Result}
\begin{itemize}
\item If F $\epsilon$ [2T, 4T] we set up a single dealership in a large town.\\
\item If F $<$ 2T, we set up 2 dealerships at the edges of a large town.\\
\item Intuition: If the sunk costs of establishing a dealership are
high, we will only set up 1.
\end{itemize}
}

\frame[plain]{
\frametitle{Downstream Moral Hazard, or Externalities from Intrabrand Competition}
Now in a small town, it is too hard to commit to not undercutting
price, since the consumers on the other side of town don't have
far to go to get to your dealership.\\
\vspace{0.2in}
In the small town case, we almost always chose a single dealership.\\
\vspace{0.2in}
Recall F $>$ 4T for the small town, so\\
\vspace{0.2in}
$\Rightarrow$ if F $<$ 2T: two dealerships\\
\hspace*{.15in} if F $>$ 2T: single dealership
}

\frame[plain]{
\frametitle{Downstream Moral Hazard, or Externalities from Intrabrand Competition}
One contractual way to get around this problem is to set up
exclusive territories so that two dealerships are simply not allowed
to compete with each other. (i.e., we contract around the problem
of intrabrand competition)\\
\vspace{0.2in}
In the previous example, each dealer becomes a monopolist on his
half of the linear market, so the profit condition now becomes:\\
\begin{center}
$\pi_{d,i}$ = B - d - F $\geq$ 0\\
\end{center}
He is contractually prohibited from stealing consumers from dealer
$j$.
}
\section{Interbrand Competition and Legal Issues}

\frame[plain]{
\frametitle{Interbrand Competition, and the use of Exclusive Dealing and Full-line Forcing}

Interbrand competition is what we have in mind in exercises that ignore retail markets: competition between different brands of cereal at the grocery store, for example.\\
\vspace{0.2in}
There are two ways in which vertical restraints may be used.
\begin{enumerate}
\item Efficiency
\item Anti-competitive foreclosure incentives upstream
\end{enumerate}
\vspace{0.2in}
The Asker paper on beer examines these issues.

}

\frame[plain]{
\frametitle{Legal Issues}

%Legal Issues
\vspace{0.1in}
\begin{itemize}
\item There are many ambiguities in the legal treatment of vertical contracts.\\
\vspace{0.1in}
\item Until 1970s, RPM and E. Territories were \underline{per se}
illegal under Sherman Act.\\
\vspace{0.1in}
\item But many states passed fair trade laws that were interpreted to
cover some of these cases.\\
\vspace{0.1in}
\item Furthermore, the Khan case in 1997 switched Maximum RPM to a ``rule of reason'' status, as did the Leegin Leather Products case in 2007 for Minimum RPM.\\
\end{itemize}
}

\frame[plain]{
\frametitle{Legal Issues}

Thus, although price fixing remains \underline{per se} illegal,
it's not always applied in vertical settings because it conflicts with
free-trade notions between mfgs and their distributors.\\
\vspace{0.2in}
Non-price issues have been generally accepted to be ok by the
courts.  Decisions turn on arguments about efficiency vs. anti-competitive effects.\\
\begin{itemize}
\item Exclusive territories 
\item Refusal to deal 
\item Foreclosure, etc.
\end{itemize}
}

\end{document}